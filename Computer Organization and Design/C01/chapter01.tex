\documentclass[12pt]{article}
\title{Chapter 1 - Computer Abstractions and Technology}
\author{Andrew Roberts}
\date{\today}
\usepackage[a4paper, total={6in, 8in}]{geometry}


\begin{document}
\maketitle
\pagenumbering{arabic}

\noindent\rule{\linewidth}{0.5mm}

\paragraph{Exercise 1.1} \textit{Aside from the smart cell phones used by a billion people, list and
describe four other types of computers.}\\

Four other types of computers include desktop computers, embedded computer systems, video game consoles, and laptops, among many others.\\

\paragraph{Exercise 1.2} \textit{The eight great ideas in computer architecture are similar to ideas
from other fields. Match the eight ideas from computer architecture, “Design for
Moore’s Law,” “Use Abstraction to Simplify Design,” “Make the Common Case
Fast,” “Performance via Parallelism,” “Performance via Pipelining,” “Performance
via Prediction,” “Hierarchy of Memories,” and “Dependability via Redundancy” to
the following ideas from other fields:\\
\textbf{a.} Assembly lines in automobile manufacturing\\
\textbf{b.} Suspension bridge cables\\
\textbf{c.} Aircraft and marine navigation systems that incorporate wind information\\
\textbf{d.} Express elevators in buildings\\
\textbf{e.} Library reserve desk\\
\textbf{f.} Increasing the gate area on a CMOS transistor to decrease its switching time\\
\textbf{g.} Adding electromagnetic aircraft catapults (which are electrically powered
as opposed to current steam-powered models), allowed by the increased power
generation offered by the new reactor technology\\
\textbf{h.} Building self-driving cars whose control systems partially rely on existing sensor
systems already installed into the base vehicle, such as lane departure systems and
smart cruise control systems}\\

\noindent
\textbf{a.} Performance via Pipelining\\
\textbf{b.} Dependability via Redundancy\\
\textbf{c.} Performance via Prediction\\
\textbf{d.} Make the Common Case Fast\\
\textbf{e.} Hierarchy of Memories\\
\textbf{f.} Performance via Parallelism\\
\textbf{g.} Design for Moore's Law\\
\textbf{h.} Use Abstraction to Simplify Design\\

\paragraph{Exercise 1.3} \textit{Describe the steps that transform a program written in a high-level
language such as C into a representation that is directly executed by a computer
processor.}\\

First, the program is compiled down from the language, such as C, to assmebly code by a compiler. From there, it is assembled by an assembler into machine code, and in C's case linked by a linker into the final binary executable.\\

\paragraph{Exercise 1.4} \textit{Assume a color display using 8 bits for each of the primary colors
(red, green, blue) per pixel and a frame size of 1280 × 1024.\\
\textbf{a.} What is the minimum size in bytes of the frame buffer to store a frame?\\
\textbf{b.} How long would it take, at a minimum, for the frame to be sent over a 100Mbit/s network?}\\

\noindent
\textbf{a.} $$\frac{8 bits}{color} \times \frac{3 colors}{pixel} \times \frac{1280 \times 1024 pixels}{frame} \times \frac{1 byte}{8 bits} = 3932160 \frac{bytes}{frame}$$\\
            $$3932160 B \times \frac{1 KB}{1024 B} = 3840 \frac{KB}{frame}$$\\
            $$3840 KB \times \frac{1 MB}{1024 KB} = 3.75 \frac{MB}{frame}$$\\
It takes 3.75 MB to store one frame of that size in a frame buffer.\\

\textbf{b.} $$\frac{1 second}{100Mbit} = \frac{1 second}{100 * 1024^{2} bit} = \frac{1 second}{104857600 bits} \times \frac{31457280 bits}{1 frame} = 0.3 \frac{second}{frame}$$\\
It would take 0.3 seconds to send one frame of that size over a 100 Mbit/second network connection.\\

\end{document}
