\documentclass{article}

\title{Chapter 1 - Introduction\\
Exercises}
\date{February 16, 2018}
\usepackage[a4paper, total={6in, 8in}]{geometry}

\begin{document}

\maketitle
\pagenumbering{arabic}

\section*{Exercises}

\subsection*{1.1 Language Processors}
\subsubsection*{Exercise 1.1.1}

The difference between a compiler and an interpreter is that a compiler translates source code to target code, whereas an interpreter executes the source code directly.

\subsubsection*{Exercise 1.1.2}

(a.) The code output by a compiler will be much faster in execution than the performance of the interpreter. \\
(b.) Since the interpreter executes statements one at a time, it is easier to debug an interpreted program since you will know on exactly which command the program fails. 

\subsubsection*{Exercise 1.1.3}

The advantage to producing assembly code versus machine code after compilation is that assembly code is still readable and debug-able by a human being, as well as being easier for the compiler to produce.

\subsubsection*{Exercise 1.1.4}

The advantages to using C as a compiler's target language is that C is a high-level language with many low-level capabilities, making it easy for another compiler to translate C code into assembly or machine code. Another advantage is that C is straightforward and terse, making it easy for a compiler to write uniform and correct translations from other sources.

\subsubsection*{Exercise 1.1.5}

An assembler needs to translate to translate assembly code into machine code. It also needs to write that machine code into standalone, relocatable modules of machine code such that a linker and loader can take that code and use it to create an executable.

\subsection*{1.2 The Structure of a Compiler}
\subsubsection*{Exercise 1.2.1}


\subsubsection*{Exercise 1.2.2}


\subsubsection*{Exercise 1.2.3}


\end{document}
